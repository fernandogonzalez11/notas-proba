\documentclass[letterpaper, 12pt]{article}
\usepackage[utf8]{inputenc}
\usepackage[margin=1.27cm]{geometry}
\usepackage{amsmath}

\title{Repaso final}
\author{Fernando González}
\date{\today}

\begin{document}
	
	\maketitle
	
	\section{Teoría}
	
	\subsection{Aspectos de teoría de conjuntos relevante a probabilidades}
	La teoría de conjuntos proporciona la base para definir eventos y espacios muestrales. Las operaciones de unión ($A \cup B$), intersección ($A \cap B$) y complemento ($A^c$) son fundamentales para describir relaciones entre eventos.
	
	\subsection{Regla de inclusión-exclusión}
	Permite calcular la probabilidad de la unión de eventos:
	$$P(A \cup B) = P(A) + P(B) - P(A \cap B)$$
	Para tres eventos:
	$$P(A \cup B \cup C) = P(A) + P(B) + P(C) - P(A \cap B) - P(A \cap C) - P(B \cap C) + P(A \cap B \cap C)$$
	
	\subsection{Espacio muestral}
	Es el conjunto de todos los posibles resultados de un experimento aleatorio, denotado como $\Omega$.
	
	\subsection{Regla de Laplace}
	Asigna probabilidades iguales a todos los resultados en un espacio muestral finito y equiprobable:
	$$P(A) = \frac{|A|}{|\Omega|}$$
	
	\subsection{Conteo: ley de la suma y del producto}
	\subsubsection{Ley de la suma}
	Si hay $n$ opciones mutuamente excluyentes para realizar una tarea, y $m$ opciones para realizar otra tarea, entonces hay $n + m$ formas de realizar una de las dos tareas.
	
	\subsubsection{Ley del producto}
	Si hay $n$ formas de realizar una tarea y $m$ formas de realizar otra tarea, entonces hay $n \times m$ formas de realizar ambas tareas.
	
	\subsection{Permutaciones (anagramas, anagramas de n letras, con objetos repetidos)}
	Permutaciones de $n$ objetos: $P(n) = n!$
	
	\noindent Permutaciones de $n$ objetos con $n_1$ objetos de tipo 1, $n_2$ objetos de tipo 2, ..., $n_k$ objetos de tipo k:
	$$P(n; n_1, n_2, ..., n_k) = \frac{n!}{n_1! n_2! ... n_k!}$$
	
	\subsection{Combinaciones (subconjuntos, distribuir r elementos iguales entre n personas)}
	Combinaciones de $n$ objetos tomados de $r$ en $r$:
	$$C(n, r) = \binom{n}{r} = \frac{n!}{r!(n-r)!}$$
	
	\subsection{Probabilidad condicional e independencia}
	Probabilidad condicional:
	$$P(A|B) = \frac{P(A \cap B)}{P(B)}$$
	Independencia: $P(A \cap B) = P(A)P(B)$ o $P(A|B) = P(A)$
	
	\subsection{Eventos excluyentes}
	$P(A \cap B) = 0$
	
	\subsection{Probabilidad total}
	$$P(A) = \sum_{i=1}^{n} P(A|B_i)P(B_i)$$
	
	\subsection{Regla de Bayes}
	$$P(B|A) = \frac{P(A|B)P(B)}{P(A)}$$
	
	\section{Ejemplos}
	
	\subsection*{Ejemplo 1}
	Se lanzan dos dados. ¿Cuál es la probabilidad de que la suma sea 7 o 11?
	
	\textit{Solución:} Hay 6 resultados posibles para cada dado, por lo que hay $6 \times 6 = 36$ resultados posibles en total. Hay 6 formas de obtener una suma de 7 y 2 formas de obtener una suma de 11. Por lo tanto, la probabilidad es $\frac{6 + 2}{36} = \frac{8}{36} = \frac{2}{9}$.
	
	\subsection*{Ejemplo 2}
	¿Cuántos anagramas se pueden formar con la palabra "MISSISSIPPI"?
	
	\textit{Solución:} La palabra tiene 11 letras, con 4 I, 4 S y 2 P. El número de anagramas es $\frac{11!}{4!4!2!} = 34650$.
	
	\subsection*{Ejemplo 3}
	Se extraen dos cartas de una baraja de 52 cartas. ¿Cuál es la probabilidad de que ambas sean ases?
	
	\textit{Solución:} Hay 4 ases en la baraja. La probabilidad de que la primera carta sea un as es $\frac{4}{52}$. Dado que la primera carta es un as, la probabilidad de que la segunda carta también sea un as es $\frac{3}{51}$. Por lo tanto, la probabilidad de que ambas cartas sean ases es $\frac{4}{52} \times \frac{3}{51} = \frac{1}{221}$.
	
	\section{Ejercicios}
	
	\begin{enumerate}
		\item ¿Cuál es la probabilidad de obtener al menos un 6 al lanzar tres dados?
		\item ¿Cuántas formas hay de distribuir 10 libros idénticos entre 4 personas?
		\item Se extraen tres cartas de una baraja de 52 cartas. ¿Cuál es la probabilidad de que las tres sean del mismo palo?
		\item Hay 10 personas en una fiesta. ¿Cuántos apretones de manos se darán si cada persona saluda a todas las demás?
		\item Una caja contiene 5 bolas rojas y 3 bolas azules. Se extraen 2 bolas al azar sin reemplazo. ¿Cuál es la probabilidad de que ambas bolas sean rojas?
	\end{enumerate}
	
\end{document}