\documentclass{article}
\usepackage{amsmath}
\usepackage{geometry}
\geometry{a4paper, margin=1in}

\begin{document}
	
	\title{Principios de Conteo}
	\author{Fernando González}
	\date{}
	\maketitle
	
	\section{Permutaciones}
	Las permutaciones cuentan de cuántas maneras se pueden ordenar $r$ elementos de un conjunto de $n$ elementos, considerando el orden:
	\begin{equation}
		P(n, r) = \frac{n!}{(n - r)!}
	\end{equation}
	
	\section{Combinaciones}
	Las combinaciones cuentan de cuántas maneras se pueden seleccionar $r$ elementos de un conjunto de $n$ elementos, sin importar el orden:
	\begin{equation}
		C(n, r) = \frac{n!}{r!(n - r)!}
	\end{equation}
	
	\section{Combinaciones con Repetición}
	Cuando los elementos pueden repetirse, el número de maneras de seleccionar $r$ elementos de un conjunto de $n$ elementos es:
	\begin{equation}
		C(n+r-1, r) = \frac{(n+r-1)!}{r!(n-1)!}
	\end{equation}
	
	\subsection{Razón}
	Se puede pensar como una secuencia de $n$ elementos indistinguibles que vamos a separar con $r - 1$ separadores (para formar $r$ grupos). Nótese, entonces, que tendremos $n + r - 1$ posiciones.
	
	Si nos olvidamos de los elementos, queremos buscar las maneras en que podemos colocar $r - 1$ separadores en nuestras $n + r - 1$ posiciones. Como 
	
	\section{Ejercicios}
	\begin{enumerate}
		\item ¿Cuántas maneras hay de ordenar 4 letras distintas de un alfabeto de 10 letras?
		\item ¿De cuántas formas se pueden elegir 5 jugadores de un grupo de 12?
		\item ¿Cuántas formas hay de distribuir 8 caramelos entre 3 niños?
	\end{enumerate}
	
\end{document}
