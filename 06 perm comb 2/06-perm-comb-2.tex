\documentclass[letterpaper, 12pt]{article}
\usepackage[utf8]{inputenc}
\usepackage[margin=1.27cm]{geometry}
\usepackage{amsmath}
\usepackage{amssymb}

\title{Técnicas adicionales de conteo}
\author{Fernando González}
\date{}

\begin{document}
	
	\maketitle
	
	\section{Teoría}
	
	\subsection{Cardinalidad del conjunto de funciones de A en B}
	
	Si $A$ y $B$ son conjuntos finitos con $|A| = m$ y $|B| = n$, entonces el número de funciones de $A$ en $B$ es $n^m$.
	
	\textbf{Explicación:} Para cada elemento de $A$, tenemos $n$ opciones en $B$. Como hay $m$ elementos en $A$, el número total de funciones es $n \times n \times \cdots \times n$ ($m$ veces), que es $n^m$.
	
	\subsection{Conteo de permutaciones con objetos repetidos}
	
	Si tenemos $n$ objetos con $n_1$ del primer tipo, $n_2$ del segundo tipo, ..., $n_k$ del $k$-ésimo tipo, donde $n_1 + n_2 + \cdots + n_k = n$, entonces el número de permutaciones distintas es:
	
	$$\frac{n!}{n_1! n_2! \cdots n_k!}$$
	
	\textbf{Explicación:} Si todos los objetos fueran distintos, tendríamos $n!$ permutaciones. Sin embargo, como tenemos objetos repetidos, debemos dividir por el número de permutaciones de cada tipo de objeto para evitar contar las mismas permutaciones múltiples veces.
	
	\subsection{Conteo de combinaciones con repetición}
	
	El número de combinaciones de $n$ objetos tomados de $r$ tipos con repetición permitida es:
	
	$$\binom{n+r-1}{r}$$
	
	\textbf{Explicación:} Podemos representar las combinaciones con repetición mediante una secuencia de $r$ estrellas (objetos) y $n-1$ barras (separadores de tipos). El número total de posiciones es $n+r-1$, y debemos elegir $r$ de ellas para las estrellas.
	
	\subsection{Soluciones naturales de $x_1 + x_2 + \cdots + x_n = r$}
	
	El número de soluciones naturales (enteros no negativos) de la ecuación $x_1 + x_2 + \cdots + x_n = r$ es:
	
	$$\binom{n+r-1}{r}$$
	
	\textbf{Explicación:} Esto es equivalente a contar el número de combinaciones con repetición, donde $x_i$ representa el número de objetos del tipo $i$.
	
	\subsection{Distribuir objetos distinguibles}
	
	Si tenemos $n$ objetos distinguibles y $k$ cajas distinguibles, el número de formas de distribuir los objetos en las cajas es $k^n$.
	
	\textbf{Explicación:} Cada objeto puede ser colocado en cualquiera de las $k$ cajas, por lo que tenemos $k$ opciones para cada uno de los $n$ objetos.
	
	\subsection{Distribuir objetos no distinguibles}
	
	Si tenemos $n$ objetos no distinguibles y $k$ cajas distinguibles, el número de formas de distribuir los objetos en las cajas es:
	
	$$\binom{n+k-1}{n}$$
	
	\textbf{Explicación:} Esto es equivalente a contar el número de combinaciones con repetición, donde los objetos no distinguibles son las estrellas y las cajas son los tipos.
	
	\section{Ejemplos}
	
	\subsection{Ejemplo 1}
	
	¿Cuántas palabras de 10 letras se pueden formar con las letras de la palabra "MATEMATICAS"?
	
	\textbf{Solución:} Tenemos 10 letras en total: 2 M, 3 A, 2 T, 1 E, 1 I, 1 C, 1 S. El número de permutaciones es:
	
	$$\frac{10!}{2!3!2!1!1!1!1!} = 151200$$
	
	\subsection{Ejemplo 2}
	
	¿Cuántas soluciones enteras no negativas tiene la ecuación $x_1 + x_2 + x_3 = 10$?
	
	\textbf{Solución:} Usando la fórmula de combinaciones con repetición:
	
	$$\binom{3+10-1}{10} = \binom{12}{10} = 66$$
	
	\subsection{Ejemplo 3}
	
	¿De cuántas maneras se pueden distribuir 7 bolas distintas en 4 cajas diferentes?
	
	\textbf{Solución:} Cada bola puede ir a cualquiera de las 4 cajas, por lo que tenemos $4^7 = 16384$ formas.
	
	\section{Ejercicios}
	
	\begin{enumerate}
		\item ¿Cuántas cadenas binarias de longitud 10 tienen exactamente 4 unos?
		\item ¿De cuántas maneras se pueden distribuir 10 bolas idénticas en 3 cajas diferentes si cada caja debe contener al menos 2 bolas?
		\item ¿Cuántas palabras diferentes se pueden formar con las letras de la palabra "PROBABILIDAD"?
		\item ¿Cuántas soluciones enteras no negativas tiene la ecuación $x_1 + x_2 + x_3 + x_4 = 20$?
		\item ¿De cuántas maneras se pueden distribuir 8 libros distintos en 5 estantes diferentes?
	\end{enumerate}
	
	
	\section{Profundización}
	\subsection{Independencia y Dependencia de Eventos}
	
	\textbf{Definición intuitiva:} Dos eventos $A$ y $B$ son \textit{independientes} si la ocurrencia de uno no afecta la ocurrencia del otro. En otras palabras, saber si $A$ ocurrió no cambia nuestra percepción de si $B$ ocurrió, y viceversa.
	
	
	\textbf{Ejemplo de independencia:} Consideremos el lanzamiento de dos monedas justas. El evento $A$ de que la primera moneda caiga en cara y el evento $B$ de que la segunda moneda caiga en cara son independientes. El resultado de la primera moneda no influye en el resultado de la segunda.
	
	
	\textbf{Ejemplo de dependencia:} Consideremos una urna con bolas rojas y azules. El evento $A$ de sacar una bola roja y el evento $B$ de sacar otra bola roja sin reemplazo son dependientes. Si sacamos una bola roja en el primer intento, la probabilidad de sacar otra bola roja en el segundo intento disminuye.
	
	\textbf{Cómo discernir independencia sin probabilidad condicional:}
	
	\begin{itemize}
		\item \textbf{Causalidad:} Si los eventos no tienen relación causal entre sí, es probable que sean independientes.
		\item \textbf{Espacio muestral:} Si el espacio muestral de un evento no se ve alterado por la ocurrencia del otro, es probable que sean independientes.
		\item \textbf{Intuición:} En muchos casos, la independencia o dependencia de eventos es intuitiva. Sin embargo, es importante ser cauteloso y no confiar únicamente en la intuición.
	\end{itemize}
	
	\textbf{Nota:} La definición formal de independencia se basa en la probabilidad condicional, que se introducirá más adelante. Sin embargo, esta explicación intuitiva proporciona una base sólida para comprender el concepto.
	
	\newpage
	
	\section*{Anexo: pregunta generadora}
	\begin{verbatim}
		Yo: Hola, soy un estudiante de Computación estudiando Probabilidades. No tengo
		problema con explicaciones técnicas y pruebas, al igual que explicaciones
		intuitivas.
		Genera un resumen del contenido de clase en un archivo .tex.
		Formato: letterpaper, márgenes de 1.27 cm
		
		Sección: Teoría
		Explicar los siguientes temas. Busca un abordamiento relativamente breve
		pero no solo poner la fórmula. Busca que la fórmula tenga algún desarrollo
		breve para llegar a ella, y de esta u otra manera poder a acercarme a cómo
		y por qué funciona esa fórmula.
		
		- Cardinalidad del conjunto de funciones de A en B
		- Conteo de permutaciones con objetos repetidos
		- Conteo de combinaciones con repetición (*)
		- (*) aplicado a soluciones naturales de x1 + x2 + ... = r
		- Distribuir objetos distinguibles
		- Distribuir objetos no distinguibles
		
		Sección: Ejemplos
		- 3 ejemplos de complejidad media que abarquen los temas de arriba.
		Explicados (no necesitan ser paso a paso) y con respuesta
		
		Sección: Ejercicios
		- 3 a 5 ejercicios de complejidad media-alta, sin respuesta. 
		
		IA: [respuesta 1]
		
		Yo: Gracias. Ahora me gustaría un snippet de TeX (no todo el archivo)
		sobre qué es y cómo se discierne independencia y dependencia de eventos.
		Nota que todavía no he sido introducido al concepto de probabilidad
		condicional, esto es solo un acercamiento. 
		
		IA: [respuesta 2]
	\end{verbatim}
\end{document}