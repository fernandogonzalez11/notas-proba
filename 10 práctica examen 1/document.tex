\documentclass[letterpaper]{article}
\usepackage[spanish]{babel}
\usepackage[utf8]{inputenc}
\usepackage{amsmath, amssymb}
\usepackage{geometry}
\geometry{
	letterpaper,
	margin=1.27cm
}

\title{Práctica de examen 1}
\author{Fernando González}
\date{\today}

\begin{document}
	\maketitle
	
	\section*{Instrucciones}
	Resuelva los siguientes problemas mostrando todo su razonamiento. Duración: 120 minutos.
	
	\begin{enumerate}
		\item \textbf{Teoría de conjuntos y probabilidad}\\
		En un experimento se lanzan tres dados equilibrados. Defina los eventos:
		\begin{itemize}
			\item $A = \{\text{La suma es par}\}$
			\item $B = \{\text{Al menos un resultado es primo}\}$
			\item $C = \{\text{Todos los dados muestran valores diferentes}\}$
		\end{itemize}
		Exprese mediante operaciones de conjuntos y calcule la probabilidad de:
		\begin{enumerate}
			\item Ocurre $A$ o $B$ pero no ambos
			\item Ocurre exactamente dos de los tres eventos
		\end{enumerate}
		
		\item \textbf{Principio de inclusión-exclusión}\\
		En una encuesta a 100 estudiantes: 45 ven anime, 68 juegan videojuegos, 35 participan en deportes, 25 ven anime y juegan videojuegos, 15 ven anime y hacen deporte, 20 juegan videojuegos y hacen deporte, y 10 hacen las tres actividades. ¿Cuántos estudiantes:
		\begin{enumerate}
			\item No realizan ninguna de estas actividades?
			\item Realizan exactamente dos actividades?
		\end{enumerate}
		
		\item \textbf{Técnica de complementos}\\
		Un sistema tiene 5 componentes independientes con probabilidad de falla $p = 0.1$ cada uno. El sistema falla si al menos 2 componentes fallan. Calcule la probabilidad de falla del sistema usando complementos.
		
		\item \textbf{Espacio muestral y Laplace}\\
		Se forman números de 4 dígitos con los dígitos 1,3,5,7,9 (sin repetición). Determine:
		\begin{enumerate}
			\item Cardinalidad del espacio muestral
			\item Probabilidad que el número sea múltiplo de 5
			\item Probabilidad que contenga exactamente dos dígitos primos
		\end{enumerate}
		
		\item \textbf{Reglas de suma y producto}\\
		La probabilidad que un día sea soleado es 0.6, lluvioso 0.3 y nublado 0.1. Si dos días consecutivos son:
		\begin{enumerate}
			\item ¿Probabilidad que haya al menos un día soleado?
			\item ¿Probabilidad que el segundo día sea diferente al primero?
		\end{enumerate}
		
		\item \textbf{Permutaciones y combinaciones}\\
		Resuelva:
		\begin{enumerate}
			\item ¿De cuántas formas pueden ordenarse 7 libros distintos si 3 específicos deben estar juntos?
			\item ¿Cuántas cadenas distintas se forman con la palabra "MATEMATICAS"?
			\item ¿Cuántos comités de 5 personas con al menos 2 mujeres pueden formarse de 6 hombres y 4 mujeres?
		\end{enumerate}
		
		\item \textbf{Distribuciones de objetos}\\
		Calcule:
		\begin{enumerate}
			\item Formas de distribuir 10 regalos diferentes a 3 niños donde cada uno recibe al menos 2 regalos
			\item Formas de colocar 15 pelotas idénticas en 4 cajas numeradas si cada caja debe contener al menos 3 pelotas
		\end{enumerate}
		
		\item \textbf{Probabilidad condicional e independencia}\\
		Se lanzan dos dados justos. Sea $A$ = "La suma es 7", $B$ = "El primer dado muestra 4". ¿Son $A$ y $B$ independientes? Demuestre.
		
		\item \textbf{Eventos excluyentes y probabilidad total}\\
		En una urna hay 5 bolas rojas, 3 azules y 2 verdes. Se extraen dos bolas sin reemplazo. Calcule:
		\begin{enumerate}
			\item Probabilidad que sean del mismo color
			\item Si la primera bola es roja, ¿probabilidad que la segunda sea verde?
		\end{enumerate}
		
		\item \textbf{Regla de Bayes}\\
		Tres máquinas producen tornillos. La máquina A produce 30\% con 2\% defectuosos, B 45\% con 3\% defectuosos, y C 25\% con 4\% defectuosos. Si se selecciona un tornillo defectuoso, ¿cuál es la probabilidad que provenga de la máquina B?
		
	\end{enumerate}
	
	\section*{Problemas adicionales (complejidad alta)}
	\begin{enumerate}
		\addtocounter{enumi}{10}
		
		\item \textbf{Combinaciones con repetición}\\
		Una pastelería ofrece 5 tipos de donas. ¿De cuántas maneras se pueden comprar 12 donas si:
		\begin{enumerate}
			\item No hay restricciones?
			\item Debe haber al menos 2 de cada tipo?
		\end{enumerate}
		
		\item \textbf{Independencia de eventos}\\
		Demuestre que si $A$ y $B$ son eventos independientes, entonces:
		\[
		P(A \cup B) = 1 - P(A^c)P(B^c)
		\]
		¿Bajo qué condiciones esta igualdad se mantiene?
		
	\end{enumerate}
	
\end{document}