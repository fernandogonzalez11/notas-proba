
\documentclass{article}
\usepackage[utf8]{inputenc}
\usepackage{amsmath}

\title{Permutaciones y combinaciones}
\author{Fernando González}
\date{}

\begin{document}

\maketitle

\section{Teoría}

\subsection{Permutaciones}
Dado un conjunto de $n$ elementos distintos, el número de formas en que se pueden ordenar todos ellos se obtiene aplicando el principio del producto y es:
\begin{equation}
    P(n) = n \times (n-1) \times \cdots \times 1 = n!
\end{equation}

Si se desea obtener las permutaciones de r elementos de un conjunto de cardinalidad n, se tiene que:
\begin{equation}
    P(n) = n \times (n-1) \times \cdots \times (n - r + 1) = \frac{n!}{(n - r)!}
\end{equation}

\subsection{Permutaciones con objetos repetidos}
Si en un conjunto de $n$ elementos hay objetos repetidos, la cantidad de permutaciones se calcula dividiendo por los factoriales de las repeticiones:
\begin{equation}
    P(n; k_1, k_2, \dots, k_m) = \frac{n!}{k_1! k_2! \dots k_m!}
\end{equation}
donde $k_i$ representa la cantidad de veces que se repite un objeto.

\subsection{Combinaciones}
Las combinaciones cuentan cuántas formas hay de seleccionar $k$ elementos de un conjunto de $n$ elementos sin importar el orden:
\begin{equation}
    C(n, k) = \binom{n}{k} = \frac{n!}{k!(n-k)!}
\end{equation}

\section{Ejemplos}

\subsection{Ejemplo 1: Permutaciones simples}
¿Cuántas formas distintas se pueden ordenar 5 libros en una estantería?

\textbf{Solución:} Como los 5 libros son diferentes, aplicamos la fórmula de permutaciones:
\begin{equation}
    P(5) = 5! = 5 \times 4 \times 3 \times 2 \times 1 = 120
\end{equation}
Respuesta: 120 formas.

\subsection{Ejemplo 2: Combinaciones y permutaciones con repetición}
Un comité de 3 personas debe seleccionarse de un grupo de 10. ¿De cuántas maneras se puede formar el comité?

\textbf{Solución:} Como el orden no importa, usamos combinaciones:
\begin{equation}
    C(10, 3) = \frac{10!}{3!(10-3)!} = \frac{10!}{3!7!} = \frac{10 \times 9 \times 8}{3 \times 2 \times 1} = 120
\end{equation}
Respuesta: 120 maneras.

\section{Ejercicios}

\begin{enumerate}
    \item ¿Cuántas formas distintas se pueden ordenar las letras de la palabra "PROBABILIDAD"?
    \item Un equipo de 4 estudiantes debe ser seleccionado de un grupo de 12. ¿Cuántas maneras hay de hacerlo?
    \item De un mazo de 52 cartas, ¿de cuántas formas se pueden elegir 5 cartas?
    \item ¿Cuántos subconjuntos de $S_{25} = \{1, 2,..., 25\}$ de 6 elementos tienen exactamente dos múltiplos de 3 y dos múltiplos de 5?
\end{enumerate}

\section{Profundización}
\subsection{Justificación de la fórmula de combinaciones}

Para contar cuántas formas hay de seleccionar $k$ elementos de un conjunto de $n$ sin importar su orden, se puede hacer lo siguiente:

\begin{enumerate}
    \item Contamos todas las formas de seleccionar y ordenar $k$ de $n$ elementos:
    \[
        P(n, k) = \frac{n!}{(n-k)!}
    \]
    \item Como el orden no importa, cada selección se contó $k!$ veces, pues esa es la cantidad de ordenamientos que tiene un conjunto singular. Como 1 conjunto representa a $k!$ ordenamientos, se tiene que:
    \[
        P(n, k) = k! \, C(n, k) \implies C(n, k) = \frac{P(n, k)}{k!} = \frac{n!}{(n-k)! \, k!}
    \]
\end{enumerate}

\section{Anexos}
\subsection{Pregunta generadora}

\begin{verbatim}
Yo:    Hola, necesito que me hagas un resumen como archivo .tex de probabilidades
    con los siguientes contenidos:

    Sección: Teoría
    - Permutaciones
    - Permutaciones con objetos repetidos
    - Combinaciones

    Sección: Ejemplos
    - 1 o 2 ejemplos que incluyan a estos temas

    Sección: Ejercicios
    - 3 ejercicios sin respuesta

IA: [respuesta 1]

Yo:    También dame un snippet de LaTeX donde explique brevemente por qué funciona
    la fórmula de combinaciones

IA: [respuesta 2]
\end{verbatim}

\end{document}