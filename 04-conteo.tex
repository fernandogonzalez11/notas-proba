
\documentclass{article}
\usepackage{amsmath}
\usepackage[utf8]{inputenc}
\usepackage{geometry}
\geometry{a4paper, margin=1in}

\begin{document}

\title{Respuestas a ejercicios de probabilidad}
\author{Fernando González}
\date{}
\maketitle


\section{Ejercicio 1}

Se tiene un conjunto \( X = \{0, 1, 2, 3, 5, 7, 8\} \) y se desean formar números de 3 dígitos distintos.

\textbf{a)} ¿Cuál es la probabilidad de que la suma de los números sea al menos 6?

Queremos calcular la probabilidad de que la suma de dos bolas sacadas esté entre 5 y 11, ambos inclusive.

El espacio muestral tiene \( 6 	\times 6 = 36 \) casos posibles (porque hay 6 bolas amarillas y 6 bolas rojas).

El conjunto de casos favorables es \( \{(2, 3), (3, 2), (1, 5), (5, 1), ...\} \).

La probabilidad de que la suma esté entre 5 y 11 es de:

\[
P = \frac{30}{36} = \frac{5}{6}
\]

\textbf{b)} ¿Cuál es la probabilidad de que al menos una bola tenga número impar?

Definimos dos eventos: \( A \) y \( B \) corresponden a que una de las bolas tiene número impar. La probabilidad es:

\[
P(A \cup B) = P(A) + P(B) - P(A \cap B)
\]

\section{Ejercicio de Ana y María}

Se tiene un grupo de 4 hombres y 3 mujeres (Ana, Lucía, María). Se elige 1 hombre y 1 mujer.

\textbf{a)} ¿Cuál es la probabilidad de que sea Ana o María?

El espacio muestral tiene 4 hombres y 3 mujeres, por lo que el total de posibles elecciones es:

\[
4 \times 3 = 12
\]

Los casos favorables son cuando la mujer elegida es Ana o María, lo que nos da 2 posibles elecciones de mujer. Así que los casos favorables son:

\[
4 \times 2 = 8
\]

Por lo tanto, la probabilidad es:

\[
P(A \cup M) = \frac{8}{12} = \frac{2}{3}
\]

\section{Ejercicio de 3 dígitos con los elementos del conjunto \( X = \{0,1,2,3,5,7,8\} \)}

El número de números de tres dígitos formados por el conjunto es de \( 6 \times 5 \times 4 = 120 \).

\textbf{a)} ¿Cuántos números de 3 cifras distintas se pueden formar que sean múltiplos de 5 o impares?

Para contar los números pares y múltiplos de 5, calculamos lo siguiente:

\begin{itemize}
    \item Los números pares tienen última cifra 0, 2 o 8, por lo que sumando los casos, obtenemos: 
    \[
    30 + 25 + 25 = 80
    \]
    \item Los múltiplos de 5 tienen última cifra 0 o 5, por lo que la cuenta es:
    \[
    30 + 25 = 55
    \]
    \item Los números que son pares y múltiplos de 5 tienen última cifra 0, y deben ser contados una sola vez. Entonces, la cuenta final usando la regla de inclusión-exclusión es:
    \[
    80 + 55 - 30 = 105
    \]
\end{itemize}

\textbf{b)} ¿Cuál es la probabilidad de que un número formado sea par o múltiplo de 5?

La probabilidad de que un número formado sea par o múltiplo de 5 es:

\[
P = \frac{105}{120} = \frac{7}{8}
\]

\section{Principio de la Suma y del Producto}

\subsection{Principio de la Suma}

Si \( A \) y \( B \) son conjuntos disjuntos, entonces la cantidad de elementos en su unión es la suma de sus cardinalidades:

\[
|A \cup B| = |A| + |B|
\]

Si los conjuntos no son disjuntos, entonces usamos la regla de inclusión-exclusión:

\[
|A \cup B| = |A| + |B| - |A \cap B|
\]

\subsection{Principio del Producto}

El número de elementos en el producto cartesiano de dos conjuntos \( A \) y \( B \) es el producto de sus cardinalidades:

\[
|A \times B| = |A| \cdot |B|
\]

En general, para \( n \) conjuntos \( A_1, A_2, \dots, A_n \):

\[
|A_1 \times A_2 \times \dots \times A_n| = |A_1| \cdot |A_2| \cdot \dots \cdot |A_n|
\]

\end{document}
