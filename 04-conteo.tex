\documentclass[letterpaper,12pt]{article}
\usepackage{amsmath}

\usepackage[margin=1in]{geometry}

\title{Fundamentos de conteo en probabilidades}
\author{Fernando González}
\date{}

\begin{document}
	
	\maketitle
	
	\section{Teoría}
	
	\subsection{Principio de la Suma}
	El principio de la suma establece que si un evento puede ocurrir de una de varias maneras, el número total de formas en que puede ocurrir el evento es la suma de las formas en que puede ocurrir cada uno de los eventos posibles, siempre y cuando los eventos sean mutuamente excluyentes. Es decir, si $A$ y $B$ son eventos mutuamente excluyentes, entonces el número total de formas en que puede ocurrir $A$ o $B$ es:
	
	\[
	|A \cup B| = |A| + |B|
	\]
	
	\subsection{Principio del Producto}
	El principio del producto se usa cuando se quiere contar el número total de formas en que pueden ocurrir dos o más eventos. Si un evento $A$ puede ocurrir de $m$ formas y un evento $B$ puede ocurrir de $n$ formas, entonces el número total de formas en que pueden ocurrir ambos eventos, uno tras otro, es el producto de las formas en que ocurre $A$ y las formas en que ocurre $B$. Es decir:
	
	\[
	|A \times B| = |A| \cdot |B|
	\]
	
	\subsection{Inclusión-Exclusión}
	El principio de inclusión-exclusión se usa para contar el número total de elementos en la unión de dos o más conjuntos, evitando contar dos veces los elementos que pertenecen a la intersección de los conjuntos. Para dos conjuntos $A$ y $B$, el número de elementos en la unión de $A$ y $B$ es:
	
	\[
	|A \cup B| = |A| + |B| - |A \cap B|
	\]
	
	Este principio se extiende a más de dos conjuntos.
	
	\subsection{Complementos}
	El complemento de un evento $A$ es el evento en el que $A$ no ocurre. La probabilidad de que no ocurra un evento $A$ es igual a 1 menos la probabilidad de que ocurra $A$. Es decir:
	
	\[
	P(A^c) = 1 - P(A)
	\]
	
	\section{Ejemplos}
	
	\subsection{Urna de bolas}
	Supongamos que tenemos una urna con 5 bolas rojas y 3 bolas azules. Si extraemos una bola al azar, ¿cuál es la probabilidad de que sea roja o azul?
	
	Usamos el principio de la suma, ya que los eventos "bola roja" y "bola azul" son mutuamente excluyentes:
	
	\[
	P(\text{roja o azul}) = P(\text{roja}) + P(\text{azul}) = \frac{5}{8} + \frac{3}{8} = 1
	\]
	
	\subsection{Suma de dos dados}
	Supongamos que lanzamos dos dados. Queremos calcular la probabilidad de que la suma de los valores sea 7. Usamos el principio del producto para contar las posibles combinaciones de resultados que sumen 7:
	
	Las combinaciones posibles son: $(1,6), (2,5), (3,4), (4,3), (5,2), (6,1)$. Hay 6 formas de obtener una suma de 7. Como hay 36 posibles resultados al lanzar dos dados, la probabilidad es:
	
	\[
	P(\text{suma de 7}) = \frac{6}{36} = \frac{1}{6}
	\]
	
	\section{Ejercicios}
	
	\begin{enumerate}
		\item En una urna hay 10 bolas rojas, 5 bolas verdes y 3 bolas azules. Si extraemos tres bolas al azar, ¿cuál es la probabilidad de que al menos una de ellas sea roja o verde?
		\item En una baraja estándar de 52 cartas, ¿cuál es la probabilidad de que al sacar dos cartas al azar, al menos una sea un as o una carta de trébol? (Recuerda que hay 4 ases en total y 13 cartas de trébol).
		\item En una encuesta, 80 personas fueron preguntadas si les gustaba el helado, el chocolate o ambos. 60 personas dijeron que les gustaba el helado, 50 dijeron que les gustaba el chocolate y 40 dijeron que les gustaba ambos. ¿Cuántas personas no dijeron que les gustaba ni el helado ni el chocolate?
	\end{enumerate}
	
\end{document}
